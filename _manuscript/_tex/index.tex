% Options for packages loaded elsewhere
\PassOptionsToPackage{unicode}{hyperref}
\PassOptionsToPackage{hyphens}{url}
\PassOptionsToPackage{dvipsnames,svgnames,x11names}{xcolor}
%
\documentclass[
]{jds}

\usepackage{amsmath,amssymb}
\usepackage{iftex}
\ifPDFTeX
  \usepackage[T1]{fontenc}
  \usepackage[utf8]{inputenc}
  \usepackage{textcomp} % provide euro and other symbols
\else % if luatex or xetex
  \usepackage{unicode-math}
  \defaultfontfeatures{Scale=MatchLowercase}
  \defaultfontfeatures[\rmfamily]{Ligatures=TeX,Scale=1}
\fi
\usepackage{lmodern}
\ifPDFTeX\else  
    % xetex/luatex font selection
\fi
% Use upquote if available, for straight quotes in verbatim environments
\IfFileExists{upquote.sty}{\usepackage{upquote}}{}
\IfFileExists{microtype.sty}{% use microtype if available
  \usepackage[]{microtype}
  \UseMicrotypeSet[protrusion]{basicmath} % disable protrusion for tt fonts
}{}
\makeatletter
\@ifundefined{KOMAClassName}{% if non-KOMA class
  \IfFileExists{parskip.sty}{%
    \usepackage{parskip}
  }{% else
    \setlength{\parindent}{0pt}
    \setlength{\parskip}{6pt plus 2pt minus 1pt}}
}{% if KOMA class
  \KOMAoptions{parskip=half}}
\makeatother
\usepackage{xcolor}
\setlength{\emergencystretch}{3em} % prevent overfull lines
\setcounter{secnumdepth}{-\maxdimen} % remove section numbering
% Make \paragraph and \subparagraph free-standing
\makeatletter
\ifx\paragraph\undefined\else
  \let\oldparagraph\paragraph
  \renewcommand{\paragraph}{
    \@ifstar
      \xxxParagraphStar
      \xxxParagraphNoStar
  }
  \newcommand{\xxxParagraphStar}[1]{\oldparagraph*{#1}\mbox{}}
  \newcommand{\xxxParagraphNoStar}[1]{\oldparagraph{#1}\mbox{}}
\fi
\ifx\subparagraph\undefined\else
  \let\oldsubparagraph\subparagraph
  \renewcommand{\subparagraph}{
    \@ifstar
      \xxxSubParagraphStar
      \xxxSubParagraphNoStar
  }
  \newcommand{\xxxSubParagraphStar}[1]{\oldsubparagraph*{#1}\mbox{}}
  \newcommand{\xxxSubParagraphNoStar}[1]{\oldsubparagraph{#1}\mbox{}}
\fi
\makeatother


\providecommand{\tightlist}{%
  \setlength{\itemsep}{0pt}\setlength{\parskip}{0pt}}\usepackage{longtable,booktabs,array}
\usepackage{calc} % for calculating minipage widths
% Correct order of tables after \paragraph or \subparagraph
\usepackage{etoolbox}
\makeatletter
\patchcmd\longtable{\par}{\if@noskipsec\mbox{}\fi\par}{}{}
\makeatother
% Allow footnotes in longtable head/foot
\IfFileExists{footnotehyper.sty}{\usepackage{footnotehyper}}{\usepackage{footnote}}
\makesavenoteenv{longtable}
\usepackage{graphicx}
\makeatletter
\def\maxwidth{\ifdim\Gin@nat@width>\linewidth\linewidth\else\Gin@nat@width\fi}
\def\maxheight{\ifdim\Gin@nat@height>\textheight\textheight\else\Gin@nat@height\fi}
\makeatother
% Scale images if necessary, so that they will not overflow the page
% margins by default, and it is still possible to overwrite the defaults
% using explicit options in \includegraphics[width, height, ...]{}
\setkeys{Gin}{width=\maxwidth,height=\maxheight,keepaspectratio}
% Set default figure placement to htbp
\makeatletter
\def\fps@figure{htbp}
\makeatother

\makeatletter
\@ifpackageloaded{caption}{}{\usepackage{caption}}
\AtBeginDocument{%
\ifdefined\contentsname
  \renewcommand*\contentsname{Table of contents}
\else
  \newcommand\contentsname{Table of contents}
\fi
\ifdefined\listfigurename
  \renewcommand*\listfigurename{List of Figures}
\else
  \newcommand\listfigurename{List of Figures}
\fi
\ifdefined\listtablename
  \renewcommand*\listtablename{List of Tables}
\else
  \newcommand\listtablename{List of Tables}
\fi
\ifdefined\figurename
  \renewcommand*\figurename{Figure}
\else
  \newcommand\figurename{Figure}
\fi
\ifdefined\tablename
  \renewcommand*\tablename{Table}
\else
  \newcommand\tablename{Table}
\fi
}
\@ifpackageloaded{float}{}{\usepackage{float}}
\floatstyle{ruled}
\@ifundefined{c@chapter}{\newfloat{codelisting}{h}{lop}}{\newfloat{codelisting}{h}{lop}[chapter]}
\floatname{codelisting}{Listing}
\newcommand*\listoflistings{\listof{codelisting}{List of Listings}}
\makeatother
\makeatletter
\makeatother
\makeatletter
\@ifpackageloaded{caption}{}{\usepackage{caption}}
\@ifpackageloaded{subcaption}{}{\usepackage{subcaption}}
\makeatother

\ifLuaTeX
  \usepackage{selnolig}  % disable illegal ligatures
\fi
\usepackage[]{natbib}
\bibliographystyle{plainnat}
\usepackage{bookmark}

\IfFileExists{xurl.sty}{\usepackage{xurl}}{} % add URL line breaks if available
\urlstyle{same} % disable monospaced font for URLs
\hypersetup{
  pdftitle={My wonderful paper},
  colorlinks=true,
  linkcolor={blue},
  filecolor={Maroon},
  citecolor={Blue},
  urlcolor={Blue},
  pdfcreator={LaTeX via pandoc}}


\title{My wonderful paper}
\author{}
\date{}

\begin{document}
\maketitle
\begin{abstract}
Unexpected results in data analysis can prompt researchers to examine
data quality and analysis steps. While some researchers can typically
diagnose issues effectively with a few checks, others may struggle to
identify appropriate checks to diagnose the problem. {[}an example of
check{]} These checks are often informal and difficult to trace and
discuss in publications, resulting in others questioning the
trustworthiness of the analysis. To address this, we formalize the
informal checks into an \emph{analysis plan} that encompasses the
analysis steps and (the unit tests): one test for whether the result
meets expectations and multiple tests for checking the analysis. We then
present a procedure to assess the quality of these unit tests based on
their accuracy and redundancy on simulated versions of the original
data. The accuracy is assessed using binary classification metrics,
\emph{i.e.}, precision and recall, while redundancy is calculated using
mutual information. This procedure can be used to conduct a sensitivity
analysis, compare different analysis plans, and to identify the optimal
cutoff point for the unit tests.
\end{abstract}

\renewcommand*\contentsname{Table of contents}
{
\hypersetup{linkcolor=}
\setcounter{tocdepth}{3}
\tableofcontents
}

\newpage

\section{Introduction}\label{introduction}

In a data analysis, an experienced researcher can often quickly assess
whether a result meets their expectation, while others may be unsure of
what to anticipate from the analysis or how to recognize when results
diverge form expected norms. (These expectations are often based on
prior knowledge, domain expertise, or common sense and they can often be
framed into a Boolean question that can be answered by a unit test. For
example, in a linear regression model, one may expect the p-value of the
coefficient to be less than 0.05. On top of the expectation on the final
results, researchers may also have expectations on the intermediate
steps of the analysis process. For example, one may expect the data
doesn't contain outliers, etc. These expectations are often checked
early on to prevent an unexpected propagate through the analysis
process.)

This expectation-setting process is often implicit and rarely documented
or discussed in publications. (There are most publication requires open
source code for publication, however, it is no instruction or guidance
on checking \ldots). Yet, making them explicit is crucial for 1) helping
junior researchers interpret results and diagnose the unexpected, 2)
providing checkpoints that support independent replication or
application of methods to new data, 3) aligning assumption across
researchers from different backgrounds.

In this paper, we formalize these informal checks into an \emph{analysis
plan} comprising the analysis steps along with the associated unit
tests. This formalization allows us to examine assumptions made about
the data during analysis and to compare different unit tests in
diagnosing unexpected outcomes. We then introduce a procedure to assess
the quality of these unit tests based on their accuracy and redundancy
across simulated version of the original data. The accuracy is assessed
using binary classification metrics -- precision and recall -- while
redundancy is measured using mutual information. We illustrate these
concepts with a simple step count example and a scenario of diagnosing
unexpected regression results when studying the effect of PM10 on
mortality.

The rest of the paper is organized as follows: Section~\ref{sec-plan}
describes the concept of analysis plan in detail.
Section~\ref{sec-examples} provides examples of analysis plan {[}more
details{]}. (need another section here or before examples?)
Section~\ref{sec-conclusion} concludes the paper.

\section{Literature review}\label{literature-review}

\subsection{Diagnosing unexpected outcomes in data
analysis}\label{diagnosing-unexpected-outcomes-in-data-analysis}

\citep{peng_diagnosing_2021} describes three pedagogical exercises of
introducing diagnosing unexpected outcome into a classroom setting.

TODO: what if the expectation is ``wrong''

\subsection{Data analysis checks in statistical
software}\label{data-analysis-checks-in-statistical-software}

Currently, little has been done on how to computationally incorporate
this diagnostic approach into data analysis workflow or software. Most
of the diagnostic tools focus on defining user-specified rules, such as
data quality checks or producing metrics to summarize model performance,
as in model diagnostics. For example, the \texttt{assertr} package
\citep{assertr} and the \texttt{pointblank} package \citep{pointblank}
provide data validation tools to allow users to define data quality
checks. In contrast, packages provide model checks tools like
\texttt{performance} \citep{performance} and \texttt{yardstick}
\citep{yardstick}, from the \texttt{tidymodels} \citep{tidymodels}
ecosystem, offers goodness-of-fit, residual analysis, and performance
metrics.

These packages provide the tools to conduct diagnostics but still don't
reflect the mental process of how data analysts diagnose the unexpected
output. For example, when an unexpected output occurs, an analyst may
check on whether a column in the data frame is between two values for
data quality. However, it is not documented what motivates the analyst
to conduct this check -- whether it also applies to other researchers
analyzing new data in the same contexts, whether it is a common practice
in the field, or whether it is a reaction to this particular data or
scenario. Currently, most of these assumptions are not discussed in the
publication or captured by tools themselves. While one might be able to
infer some of the mental process of the analysts from external sources,
such as talking to them or watching screencast videos produced by the
analysts e.g.~TidyTuesday screencast videos, these insights are not
systematically documented or made machine-readable. This gap highlights
the need for tools that provide higher level documentation of reasoning
behind the checks, facilitating a more transparent and interpretable
analysis process.

\subsection{Unit tests}\label{unit-tests}

Why develop unit tests in software engineering? Because it is difficult
to predict the outcome of a program, especially when the program is
complex. Because the loss associated with a program failure is costly.

Software testing relies on ``oracles'' to provide the expected output
necessary for verifying test results. For example, to test whether the
program correctly calculates 1 + 1, one need to supply the correct
answer, 2. However, establishing this ``correct'' output can sometimes
be challenging, where obtaining a solution may be difficult without the
program itself. This situation leads to the oracle problem
\citep{barr2014oracle}. In data analysis, the similar oracle problem can
happen, as the ``truth'' of an outcome, the expectation, depends on the
underlying theory or interpretation. For example, in a linear regression
model, the significance of a coefficient may be expected or unexpected
based on the theory, making it challenging for researcher with a
different theory to assess the results and the analysis.

Existing tools for checks in data analysis including those check for
data quality and model diagnostics. In software development,

\begin{itemize}
\tightlist
\item
  the \texttt{testthat} package \citep{testthat} provides the
  infrastructure for testing R code. It allows users to write unit tests
  for functions and packages.
\item
  The \texttt{assertthat} package \citep{assertthat} helps to write
  assertion statements that are supposed to be true at a certain point
  in the code for defensive programming.
\end{itemize}

\section{Analysis plan}\label{sec-plan}

\subsection{Framing checks into unit
tests}\label{framing-checks-into-unit-tests}

Q: Whether we should formulate these concept with math notation?? A:
only if it helps

An analysis plan is a set of analysis steps combined with expectations.
Expectations represent our belief about certain aspects of the analysis,
independent of the analysis itself. It can be divided into two types:
\emph{outcome expectation} and \emph{plan expectation}. Outcome
expectation refers to what we anticipate from the main result of the
analysis based on prior knowledge. They shape how we interpret the
results and assess whether they are consistent with existing knowledge
or indicate the need for updates \citep{grolemund_cognitive_2014}. For
example, in public health, prior research shows the average increase in
mortality rate per unit increase in PM10 is about 0.50\%
\citep{liu2019ambient}. This serves as an expectation for similar future
studies. Plan expectations concern the intermediate steps within the
analysis rather than the final outcome. They serve as checkpoints to
detect deflection in the analysis process For example, we may expect
temporal data to be ordered with no gaps and duplicates, or expect that
temperature will be a significant covariate in the linear regression
model of PM10 on mortality.

(might be useful) Analysis plans can be constructed at various
granularities, at the highest level, one may only has a plan of the
specific method used for analysing data and the expected outcome. This
provides little guidance when a deviation from expectation occurs. At
the lowest level, one may have a plan for each data entry and every data
handling steps. This provides too much detail and may not be practical
in practice.

Experienced analysts often have implicit expectation about the outcome
and rely on a few ``directional signs'' to check when the outcome
deviate from those expectation. However, these expectations are rarely
made explicit within the analysis workflow. This makes it challenging
for consumers of the analysis to evaluate the results, since it becomes
difficult to disentangle whether discrepancies arise from differing
expectations or from the use of statistical technique, without running
the analysis themselves. Non-expert analysts, lacking prior knowledge or
instinct, may not have clear expectations of the results. This can lead
to reduced confidence of the analysis and makes it more difficult and
time-consuming to diagnose the cause of the deviation when the results
don't align with expectations. By explicitly formulating these
expectations, an analysis plan can guide the analysis process,
facilitate the communication and evaluate the validity of the results.

The expectations can be thought of as a set of unit tests used to
validate the results of data analysis. By specifying a range of values
for these tests, multiple versions of the dataset can be generated to
satisfy different sets of plan expectations. This allows us to present
what we called the ``result universe'' -- the complete set of possible
results that can be obtained from one data analysis process. By
visualizing the result universe, data analysis consumers can observe how
changes in expectations affect the results and the range of alternative
outcomes that could arise under different conditions. This enables them
to evaluate the outcomes based on their own plan expectations and gain a
broader perspective on how the actual results produced by analysts fit
within this spectrum of possibilities, promoting transparency and trust
in analysis.

Furthermore, by generating multiple versions of the data, we can emulate
various scenarios within the same context for students to exercise
judgement when conducting data analysis in a classroom setting.

\subsection{A toy example}\label{a-toy-example}

Let's think about a 5-day step count. You make a resolution to walk on
average 5000 steps a day (your expectation) and using an app to record
your step count. After 5 days, the app tells you've walked on average
8000 steps.

It is easy to come up with reasons why an 8000 average step is resulted
based on common sense:

\begin{enumerate}
\def\labelenumi{\arabic{enumi}.}
\tightlist
\item
  you may run a 10k on day 1, resulting a high step count on the day
  (outlier on the right).
\item
  you left your phone at home on day 3, resulting a zero or minimal step
  count on the day (outlier on the left).
\item
  you may realise the step count may increase since you were in a hiking
  trip in the last five days (average shift).
\end{enumerate}

Based on these reasons, you may devise a set of unit tests to check the
step count data, i.e.~check the maximum and minimum step count, check
the difference between each day.

\begin{itemize}
\tightlist
\item
  If the daily count looks like c(4000, 5000, 5500, 5500, 20000), the
  maximum check will flag the data for investigate the maximum. The
  difference between days test will also flag the data
\item
  If the daily count looks like c(20000, 20000, 20000, 20000, 20000),
  the maximum check will flag the data for investigate the maximum.
\end{itemize}

Some part of the space is impossible: c(0, 4000, 5000, 5500, 5500) is
flagged by the minimal tests but won't cause an average of 8000 average
step.

The statistical procedure of averaging 5 numbers ``around 5000'' to get
a mean of 5000 is \emph{consistent} meaning if all the numbers are
around 5000, we are guaranteed to get a mean around 5000. We could
devise 5 unit tests to check each number. Since you're more familiar
with your daily life, you may realise the step count may increase since
you were in a hiking trip in the last two days. This may prompt you to
check the step count.

In a data analysis, it is not practical to check every entry of the
data, a similar strategy of devising tests to check for

\begin{itemize}
\tightlist
\item
  The combination of unit tests are not unique
\item
  The unit tests provide guidance for diagnosing the results, but are
  not red flags: c(2000, 2000, 5000, 8000, 8000) will likely to fail the
  max diff test but receive a within expectation mean.
\end{itemize}

\section{Method}\label{method}

\subsection{A workflow to assess the quality of unit
tests}\label{a-workflow-to-assess-the-quality-of-unit-tests}

In real-world applications, it is rare to create a set of unit tests can
fully guarantee expected results. On one hand, there is the cost of
effort involved in manually developing all these tests; on the other,
there is the inherent complexity of the problem. (can we - if we're
thriving for detecting 95\% of the cause?). However, the quality of unit
tests can be evaluated by simulated data. One set of tests is considered
better than another if a small set of tests can reliably detect
unexpected outcomes, which brings two criteria in the evaluation metric:
accuracy and parsimony.

Accuracy refers to a set of tests' ability to accurately detect
unexpected outcomes while minimizing false positives and false
negatives. A false positive can indicate (caution or skepticism on
checking the data), whereas a false negative suggests the tests may lack
sensitivity to unexpected outcomes. Since both plan and outcome
expectations can be framed as unit tests with binary outcomes, one
approach is to predict the outcome expectation based on the plan
expectation. Performance of the tests can then be evaluated using
precision and recall metrics from the confusion matrix:

\begin{itemize}
\tightlist
\item
  precision: the proportion of correctly identified unexpected results
  (true positives) out of all the predicted unexpected results (true
  positives + false positives)
\item
  recall: the proportion of correctly identified unexpected results
  (true positives) out of all the actual unexpected results (true
  positives + false negatives)
\end{itemize}

The second criteria is parsimony in the tests. While tests may score
high on accuracy, they may be less effective at explaining the reasons
behind unexpected results. This could happen if a set of tests are all
tangentially related to the cause of the unexpected results, but none
addressing the root cause. It may also occur if the tests are correlated
with one another, leading to redundancy.

\begin{itemize}
\item
  explain mutual information
\item
  explain logic regression
\item
  explain combining precision, recall, and independence together through
  ``means''
\item
  A logic regression (ref) is used to model the relationship between the
  plan and outcome expectations. (justify the use of Logic regression)
\end{itemize}

\subsection{Toy example revisited}\label{toy-example-revisited}

\begin{itemize}
\item
  provide interpretation at different scenarios:

  \begin{enumerate}
  \def\labelenumi{\arabic{enumi})}
  \tightlist
  \item
    one test is flagged, the prediction is as expected:
  \item
    multiple tests are flagged, the prediction is unexpected,
  \item
    no test is flagged, the prediction is unexpected,
  \item
    no test is flagged, the prediction is as expected
  \end{enumerate}
\end{itemize}

\section{Applications}\label{sec-examples}

Three examples are presented to illustrate how the concept of analysis
plan can be applied to data analysis. {[}toy example{]}.
Section~\ref{sec-linear-reg} illustrates how constructing the result
universe in a linear regression model of PM10 on mortality can help
understand the impact of sample size, model specification, and variable
correlation structure on data analysis. {[}example three{]}

\subsection{Linear regression}\label{sec-linear-reg}

Consider a linear regression model to study the effect of PM10 on
mortality (provide context of using PM10 to study mortality). Analysts
may expect a significant (p-value \(\le\) 0.05) PM10 coefficient in the
linear model from the literature. This is the \emph{outcome
expectation}. There are multiple factors that can affect the outcome
expectation of linear regression, which here is called \emph{plan
expectation}, for example, 1) sample size, 2) model specification, and
3) correlation structure between variables. Adequate sample size is
required to achieve the desired power to detect the significance of PM10
on mortality. Temperature is often an important confounder to consider
in such study (add reference). From some domain knowledge, an analyst
may expect that the significance of PM10 coefficient can be attained by
adding temperature to the model. Analysts may also expect certain
correlation structure between PM10, temperature, and mortality, and the
distribution of each variable.

To build the result universe, datasets can be simulated to either meet
and fail these plan expectations, allowing the analysts to observe the
significance of PM10 coefficient. Here, sample sizes of 50, 100, 500,
and 1000 are considered. Two model specifications are included: 1)
linear model with PM10 as the only covariate
(\(\text{mortality} \sim \text{PM10}\)), 2) linear model with PM10 and
temperature as covariates
(\(\text{mortality} \sim \text{PM10} + \text{temp}\)). A grid-based
approach is used to simulate correlation structure. Reasonable ranges of
correlation between the three variables are
\(\text{cor}(\text{mortality}, \text{PM10}) \in [-0.01, 0]\),
\(\text{cor}(\text{mortality}, \text{temperature}) \in [-0.6, -0.2]\),
and \(\text{cor}(\text{PM10}, \text{temperature}) \in [0.2, 0.6]\).

\begin{itemize}
\item
  add a paragraph to describe the simulation process
\item
  add a fourth panel to describe the comparison of a right/ wrong
  expectation, i.e.~correlation on PM10 and mortality
\end{itemize}

Figure~\ref{fig-result-universe} shows that result universe of the
linear regression model and how a change of decision in one of the plan
expectations above affect the outcome expectation. Panel a) is colored
by the outcome expectation -- whether a significant p-value is found in
the PM10 coefficient. Panel b) shows the effect of adding temperature to
the model and the results show that the significance of PM10 coefficient
can be achieved by adding temperature to the model for a sample size of
500. Panel c) shows that increasing sample size from 50 to 100 enhances
the significance of p-value for PM10 and the significance remains with
further increases in sample size. {[}note: weave the ``actual data''
into the example linear regression model{]}

\textsubscript{Source:
\href{https://huizezhang-sherry.github.io/paper-analysis-plan/index.qmd.html}{Article
Notebook}}

\phantomsection\label{cell-fig-result-universe}
\begin{figure}[H]

\centering{

\includegraphics{index_files/figure-pdf/fig-result-universe-1.pdf}

}

\caption{\label{fig-result-universe}The result universe of linear
regression model to study the effect of PM10 on mortality: a) colored by
whether the p-value of PM10is significant (less than 0.05), b) the
effect of adding temperature to the model for a sample size of 500, c)
the effect of increasing sample size for a fixed correlation structure.}

\end{figure}%

\textsubscript{Source:
\href{https://huizezhang-sherry.github.io/paper-analysis-plan/index.qmd.html}{Article
Notebook}}

\section{Discussion}\label{discussion}

\begin{itemize}
\item
  how to systematically simulate data is still unknown, sensitivity of
  the simulation to the results
\item
  plotting is a critical way to check data and they can still be frame
  into a unit test. it is a open problem to how to encode the
  visualization into the unit tests. Maybe a procedure like confirm plot
  (this looks alright to you) and then press the button to continue
\item
  currently no automated way to generate unit tests. It is interesting
  to see the automation of generating unit tests, although it requires
  the inputs from experts across a wide array of common scenarios.
\end{itemize}

\section{Conclusion}\label{sec-conclusion}


\renewcommand\refname{References}
  \bibliography{references.bib}



\end{document}
